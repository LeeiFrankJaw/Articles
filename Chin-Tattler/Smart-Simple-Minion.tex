\documentclass[a5j,12pt]{ltjtarticle}

\title{\gtfamily 聰明人和傻子和奴才 \\[1em]
\hfil\normalsize\mcfamily ——野草之二十\hspace*{-3\zw}}
\author{\small\kai 魯迅}
\date{}

\usepackage{luatexja-fontspec}
\setmainjfont[
TateFeatures={JFM=custom/{quanjiao,vert,hwcl}},
  AltFont={{Range="FF1A-"FF1B,Font=Adobe Song Std}}
]{FZNew XiuLiB-Z11}
\setsansjfont{FZHeiB-B01}
\newjfontfamily\kai{FZKaiB-Z03}

\parindent2\zw

\begin{document}
\maketitle

奴才總不過是尋人訴苦,也只能這樣,只要這樣。有一日,他遇到一個聰明人。

「先生!」他悲哀地說,眼淚聯成一線,就從眼角上直流下來。「你知道的。我所過的簡直不是人的生活。吃的是一天未必有一餐,這一餐又不過是高粱皮,連豬狗都不要吃的,且只有一小碗……。」

「這實在令人同情。」聰明人也慘然說。

「可不是麼!」他高興了。「可是做工是晝夜無休息的:清早擔水晚燒飯,上午跑街夜磨麵,晴洗衣裳雨張傘,冬燒汽爐夏打扇,半夜要煨銀耳,侍候主人耍錢,頭錢從來沒分,有時還挨皮鞭……。」

「唉唉……。」聰明人嘆息著,眼圈有些發紅,似乎要下淚。

「先生!我這樣是敷衍不下去的。我總得另外想法子。可是什麼法子呢?……」

「我想,你總會好起來……。」

「是麼?但願如此。可是我對先生訴了冤苦,又得了你的同情和慰安,已經舒坦得不少了。可見天理沒有滅絕……。」

\bigskip

但是,不幾日,他又不平起來了,仍然尋人去訴苦。

「先生!」他流著淚說,「你知道的。我住的簡直比豬窩還不如。主人並不將我當人;他對他的叭兒狗還要好到幾萬倍……。」
\end{document}

% Local Variables:
% TeX-engine: luatex
% End:
