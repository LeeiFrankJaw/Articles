\documentclass[a5j,12pt]{ltjtarticle}

\title{\gtfamily 匆匆}
\author{\small\kai 朱自清}
\date{}

\usepackage{luatexja-fontspec}
\setmainjfont[TateFeatures={JFM=custom/{kaiming,vert,hwcl}}]{Adobe Song Std}
\setsansjfont{Adobe Heiti Std}
\newjfontfamily\kai{Adobe Kaiti Std}

\parindent2\zw

\usepackage{ulem}
\showthe\ULdepth
\makeatletter
\UL@protected\def\kahasen{%
    \leavevmode \bgroup 
    \markoverwith{\lower5\p@\hbox{\sixly \char58}}\ULon}
\makeatother

\begin{document}
\maketitle

燕子去了,有再來的時候;楊柳枯了,有再青的時候;桃花謝了,有再開的時候。但是,聰明的,你告訴我,我們的日子爲什麽一去不復返呢?——是有人偷了他們罷:那是誰?又藏在何處呢?是他們自己逃走了罷:現在又到了那裏呢?

我不知道他們給了我多少日子;但我的手確乎是漸漸空虛了。在默默裏算着,八千多日子已經從我手中溜去;像針尖上一滴水滴在大海裏,我的日子滴在時間的流裏,没有聲音,也没有影子。我不禁頭涔涔而淚潸潸了。

去的儘管去了,來的儘管來着;去來的中間,又怎樣地匆匆呢?早上我起來的時候,小屋裏射進兩三方斜斜的太陽。太陽他有腳啊,輕輕悄悄地挪移了;我也茫茫然跟着旋轉。於是——洗手的時候,日子從水盆裏過去;吃飯的時候,日子從飯碗裏過去;默默時,便從凝然的雙眼前過去。我覺察他去的匆匆了,伸出手遮挽時,他又從遮挽着的手邊過去。天黑時,我躺向床上,他便伶伶俐俐地在我身上跨過,從我腳邊飛去了。等我睜開眼和太陽再見,這算又溜走了一日。我掩着面嘆息。但是新來的日子的影兒又開始在嘆息裏閃過去。

在逃去如飛的日子裏,在千門萬户的世界裏的我,能做些什麽呢?只有徘徊罷了,只有匆匆罷了;在八千多日的匆匆裏,除徘徊外,又剩些什麽呢?過去的日子如輕煙,被微風吹散了,如薄霧,被初陽蒸融了;我留着些什麽痕跡呢?我何曾留着像游絲樣的痕跡呢?我赤裸裸來到這世界,轉眼間也將赤裸裸回去罷?但不能平的,爲什麽偏要白白走這一遭啊?

你聰明的,告訴我,我們的日子爲什麽一去不復返呢?

\hfill \date{一九二二年三月二十八日,台州}

\hfill (原載自\kahasen{時事新報・文學旬刊}第三十四期)\hspace{2\zw}
\end{document}

% Local Variables:
% TeX-engine: luatex
% End:
